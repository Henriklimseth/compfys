\documentclass[norsk, 12pt]{article}

\usepackage[T1]{fontenc}    % Riktig fontencoding
\usepackage[utf8]{inputenc} % Riktig tegnsett
\usepackage{babel}          % Ordelingsregler, osv
\usepackage{graphicx}       % Inkludere bilder
\usepackage{booktabs}       % Ordentlige tabeller
\usepackage{url}            % Skrive url-er
\usepackage{textcomp}       % Den greske bokstaven micro i text-mode
%\usepackage{units}          % Skrive enheter riktig
\usepackage{float}          % Figurer dukker opp der du ber om
%\usepackage{lipsum}         % Blindtekst
\usepackage{amsmath}        % Mattestæsj
\usepackage{listings}       % Kodetekst
%\usepackage{lipsum} 	    % Package to generate dummy text throughout this template
%\usepackage[lined,boxed,commentsnumbered]{algorithm_2e}
\usepackage{multicol} 	    % Used for the two-column layout of the document
\usepackage[a4paper,margin=1.0in]{geometry}
%\usepackage{comment}
\usepackage{amsfonts}
\usepackage{parskip}

% Egendefinerte kommandoer
\newcommand{\f}{\frac}
\newcommand{\ddr}{\frac{d}{dr}}
\newcommand{\vr}{\vec{r}}

% JF i margen
\makeatletter
\renewcommand{\subsubsection}{\@startsection{subsubsection}{3}{-2cm}%
{-\baselineskip}{0.5\baselineskip}{\bf\large}}
\makeatother
\newcommand{\jf}[1]{\subsubsection*{JF #1}\vspace*{-2\baselineskip}}

% Skru av seksjonsnummerering
\setcounter{secnumdepth}{-1}

%, trim = 1cm 7cm 1cm 7cm % PDF-filer som bilde

\begin{document}

% Forside
\begin{titlepage}
\begin{center}

\textsc{\Large FYS4150 - Computational Physics}\\[0.5cm]
\rule{\linewidth}{0.5mm} \\[0.4cm]
{ \huge \bfseries  PROJECT 2}\\[0.10cm]
\rule{\linewidth}{0.5mm} \\[1.5cm]
\textsc{\Large }\\[1.5cm]
\textsc{}\\[1.5cm]

% Av hvem?
%\begin{minipage}{0.49\textwidth}
 %   \begin{center} \large
  %      Henrik Sverre Limseth\\ \url{henrisli@uio.no} \\[0.8cm]
   % \end{center}
%\end{minipage}
%\bigskip \\
\begin{minipage}{0.49\textwidth}
    \begin{center} \large
        Henrik Sverre Limseth\\ \url{henrisli@uio.no} \\[0.8cm]
    \end{center}
\end{minipage}

%\vfill

% Dato nederst
\large{Dato: \today}

\end{center}
\end{titlepage}

\section{Sammendrag}
I dette prosjektet skal korrelasjonsenergien mellom to elektroner i et heliumatom beregnes. Problemet ender i et 
seksdimensjonalt integral som kan løses analytisk. Integralet blir deretter løst med forskjellige numeriske metoder:
Gauss-Legendre kvadratur, Gauss-Laguerre kvadratur og Monte-Carlo metoder med og uten viktighetsprøvetaking (``importance
sampling'').


All kode som har blitt brukt kan finnes på githubadressen:

https://github.com/Henriklimseth/compfys/tree/master/Project3

\section{Problemet}

For å beregne korrelasjonsenergien mellom to elektroner i et heliumatom må forventningsverdien av potensialet mellom
dem regnes ut. Første steg er å finne en tilnærming til bølgefunksjonen til elektron-elektronsystemet. Vi antar begge
elektronene i laveste orbital. Dermed må de to elektronene ha motsatt spinn og den romlige bølgefunksjonen må være
symmetrisk med hensyn til ombytte av de to elektronene. For ett elektron i denne tilstanden er bølgefunksjonen
$$\psi(\vr)\propto e^{-\alpha r}$$
med $r = |\vr| = \sqrt{x^2+y^2+z^2}$ og $\alpha$ er en parameter som skal svare til ladningen til heliumkjernen:
$\alpha = 2$.

Vi gjetter på at topartikkelbølgefunksjonen er produktet av to ènpartikkelbølgefunksjoner:

$$\Psi (\vr_1, \vr_2) = e^{-2(r_1+r_2)}$$
(ikke normalisert).

Siden Coloumbpotensialet $V(\vr_1, \vr_2) \propto \f{1}{|\vr_1-\vr_2|}$ vil størrelsen av interesse være
$$\left\langle\f{1}{|\vr_1-\vr_2|}\right\rangle = \left\langle\Psi\left|\f{1}{|\vr_1-\vr_2|}\right|\Psi\right\rangle
= \int_{\mathbb{R}^3}d\vr_1d\vr_2e^{-4(r_1+r_2)}\f{1}{|\vr_1-\vr_2|}.$$

Dette integralet kan løses analytisk ved hjelp av et snedig triks: Vi bruker sfæriske koordinater:
$d\vr = r^2\sin\theta drd\theta d\phi$
og integrerer først over $\vr_2$ (mens vi holder $\vr_1$ konstant) og velger koordinatsystem slik at
$z_2$-aksen er parallell med $\vr_1$. På denne måten blir vinkelen mellom $\vr_1$ og $\vr_2$ lik $\theta_2$ og 
$$|\vr_1-\vr_2| = \sqrt{r_1^2+r_2^2-2r_1r_2\cos\theta_2}.$$
Vi får:

\begin{align*}
 \left\langle\f{1}{|\vr_1-\vr_2|}\right\rangle&=2\pi\int d\vr_1\left\{\int\limits_0^\infty\int\limits_0^\pi dr_2d\theta_2e^{-4(r_1+r_2)}\f{r_2^2\sin\theta_2}{\sqrt{r_1^2+r_2^2-2r_1r_2\cos\theta_2}} \right\} \\
 &= 2\pi\int d\vr_1\left\{\int\limits_0^\infty dr_2e^{-4(r_1+r_2)}\f{r_2}{r_1}\left(\sqrt{r_1^2+r_2^2+2r_1r_2}-\sqrt{r_1^2+r_2^2-2r_1r_2}\right)\right\}\\
 &= 2\pi\int d\vr_1\left\{\int\limits_0^\infty dr_2e^{-4(r_1+r_2)}\f{r_2}{r_1}\left(|r_1+r_2|-|r_1-r_2|\right)\right\}\\
 &= 4\pi\int d\vr_1e^{-4r_1}\left\{\f{1}{r_1}\int\limits_0^{r_1}dr_2e^{-4r_2}r_2^2 + \int\limits_{r_1}^\infty dr_2e^{-4r_2}r_2\right\}\\
 &= 4\pi\int d\vr_1\left\{\f{1}{32r_1}e^{-4r_1}-\f{1}{32r_1}e^{-8r_1}-\f{1}{16}e^{-8r_1}\right\}\\
 &= 4\pi\int\limits_0^\infty\int\limits_0^\pi\int\limits_0^{2\pi}d\phi d\theta dr\sin\theta\left(\f{1}{32}re^{-4r}-\f{1}{32}re^{-8r}-\f{1}{16}r^2e^{-8r}\right)\\
 &= \f{\pi^2}{2}\int\limits_0^\infty dr\left(re^{-4r}-re^{-8r}-2r^2e^{-8r}\right)\\
 &= \f{\pi^2}{2}\left(\f{1}{16}-\f{1}{64}-\f{1}{128}\right)\\
 &= \f{5\pi^2}{16^2}.
\end{align*}

Så vi har noe å sammenlikne de numeriske resultatene med.

\section{Numeriske metoder}
\subsection{Gaussisk kvadratur}
Vi tilnærmer et integral
$$ I = \int_a^bf(x)dx \approx \sum\limits_{i=1}^Nw_if(x_i)$$
I integrasjonsmetoder basert på Taylorrekker vil man ut ifra antall punkter man vil bruke ($N$) bestemme enhver $x_i$
ved at $x_i = a + \f{b-a}{N}\cdot i$, mens $w_i$ bestemmes ut ifra hvilken metode man bruker.

Med Gaussisk kvadratur vil man bestemme både $x_i$ og $w_i$ optimalt ved at man ikke krever uniformt fordelte $x_i$
mellom $a$ og $b$. Dermed får man $2N$ parametere å jobbe med i  Gaussisk kvadratur, i motsetning til $N$ i Taylorrekkemetoder.
Dette medfører at mens Taylorrekkemetoder integrerer nøyaktig polynomer av orden $N-1$ kan metoder basert på
Gaussisk kvadratur integrere nøyaktig polynomer av orden $2N-1$, selv om vi bare jobber med $N$ $x$-verdier. Dette
medfører større nøyaktighet siden et høyere grads polynom bedre vil kunne tilnærme funksjonen vi ønsker å integrere:
$$\int f(x)dx\approx \int P_{2N-1} = \sum\limits_{i=1}^{N} P_{2N-1}(x_i)w_i$$
er mer nøyaktig enn
$$\int f(x)dx\approx \int P_{N-1} = \sum\limits_{i=1}^{N} P_{N-1}(x_i)w_i.$$

Ideen videre bygger på bruk av ortogonale polynomer for å bestemme vektene og punktene.

\subsection{Gauss-Legendre}
Legendrepolynomene $L_k(x)$ er ortogonale polynomer definert på $[-1, 1]$. De er løsninger av differensiallikningen
$$C(1-x^2)f(x)+(1-x^2)\f{d}{dx}\left((1-x^2)\f{df}{dx}\right)=0$$
hvor $C$ er en konstant og er gitt ved Rodrigues' formel:
$$L_k(x) = \f{1}{2^kk!}\f{d^k}{dx^k}(x^2-1)^k,\ \ \ \ \ k\in\mathbb{N}$$
med $L_0(x) = 1$.

Ortogonaliteten er gitt ved:
$$\int\limits_{-1}^1 L_i(x)L_j(x)dx = \f{2}{2i+1}\delta_{ij}$$

Disse polynomene kan brukes for å finne vekter og integrasjonspunkter (for integraler mellom -1 og 1) på følgende måte:
Vi tilnærmer $f(x) \approx P_{2N-1}(x)$. Ved en polynomdivisjon på $L_N(x)$ gir dette
$$P_{2N-1}(x) = L_N(x)P_{N-1}(x)+Q_{N-1}(x)$$

hvor $P_{N-1}$ og $Q_{N-1}(x)$ er polynomer av grad $\leq N-1$.

Siden $\left\{L_k(x)\right\}_{k=0}^{k=N-1}$ er ortogonale spenner de ut det $(N-1)$'te polynomrommet, så vi kan skrive

\begin{align}
 P_{N-1}(x) &= \sum\limits_{j=0}^{N-1}\beta_j L_j(x)\\
 Q_{N-1}(x) &= \sum\limits_{i=0}^{N-1}\alpha_i L_i(x)
\end{align}

Siden $L_0(x) = 1$ og ved ortogonalitetsrelasjonen får vi dermed

$$\int\limits_{-1}^1 P_{2N-1}(x) = 2\alpha_0$$

og hele integrasjonsproblemet er redusert til å finne koeffisienten $\alpha_0$ for $Q_{N-1}(x)$ uttrykt i
Legendrepolynombasis.

Videre har vi at ved nullpunktene til $L_N(x)$, $\left\{x_k\right\}_{k=0}^{N-1}$ må
$P_{2N-1}(x_k) = Q_{N-1}(x_k)$, altså er disse verdiene for $Q_{N-1}(x)$ kjente.
Vi får dermed et likningssett for $\left\{\alpha_i\right\}$:
$$Q_{N-1}(x_k) = \sum\limits_{i=0}^{N-1}\alpha_i L_i(x_k) = \sum_{i=0}^{N-1}\alpha_i L_{ik},\ \ \ k = 0,1,\dots ,N-1$$

som kan omskrives til matriselikningen
$$\vec Q = L\vec\alpha \Leftrightarrow \vec \alpha = L^{-1}\vec Q$$
hvor $L$ er matrisa med $L_ik$ som elementer, $\vec Q$ er vektoren med $Q_{N-1}(x_k)$ som elementer og
$\vec \alpha$ er vektoren med $\alpha_i$ som elementer. Siden polynomene $L_i$ er ortogonale kan ingen av kolonnene
i $L$ være lineært avhengige av de andre, så $L$ er invertibel.

Fra matriselikningen er det klart at $\alpha_0 = \sum\limits_{i=0}^{N-1}(L^{-1})_{0i}Q_{N-1}(x_i)$, og dermed må
$$\int\limits_{-1}^1f(x)dx \approx 2\sum\limits_{i=0}^{N-1}(L^{-1})_{0i}P_{2N-1}(x_i).$$
Så vi bestemmer vektene som $w_i=2(L^{-1})_{0i}$ og integrasjonspunktene som nullpunktene til $L_N(x)$, $x_i$.

For å anvende Gauss-Legendre kvadraturen på et generelt intervall $[a,b]$ utfører vi bare substitusjonen:
$$t = \f{b-a}{2}x+\f{b+a}{2}$$ slik at
$$\int\limits_a^bf(t)dt = \f{b-a}{2}\int\limits_{-1}^1f(t(x))dx.$$

I vårt tilfelle er grensene $\pm\infty$, men siden integranden dør raskt ut for økende avstand fra origo velger vi bare
å integrere innenfor et stort nok endelig intervall. Funksjonen $gauleg$ i biblioteket $lib.cpp$ ble brukt for å regne
ut alle integrasjonspunkter og vekter.

Vi løser det seksdimensjonale integralet i kartesiske koordinater med Gauss-Legendre kvadratur. Steg èn er å finne
verdier som kan tilnærmes med uendelig, så vi studerer når integranden er tilnærmet null:
$$e^{-4(r_1+r_2)} \sim e^{-8r} = 10^{-8} \Rightarrow r = \ln10\approx2.3$$


\begin{figure}
 \centering
 \includegraphics[scale=0.7]{integrandplot.png}
 \caption{Omtrentlig størrelse på integranden som funksjon av avstand fra origo for èn koordinat.
 Punktet $x=2.3$ er merket av.}
\end{figure}

Som vi ser av figur 1 er $2.3\approx\infty$ en gyldig tilnærming. 

Deretter brukes funksjonen $gauleg$ til å finne vekter og integrasjonspunkter og integralet beregnes ved en
seksdobbel sum, en over hver koordinat/vekt.
Tabell 1 viser resultatene fra denne metoden.

\begin{table}
 \centering
 \begin{tabular}{|c|c|c|c|}\hline
 n &Beregning &Relativ feil &Tid [s] \\ \hline
 10      &0.11204       &0.418774      &0.18\\
 15      &0.208919      &0.0837953      &1.91\\
 20      &0.172009      &0.107678      &10.65\\
 25      &0.190286      &0.0128633      &40.24\\
 30      &0.183952      &0.0457216      &119.88\\ \hline
  
 \end{tabular}
\caption{Resultater for Gauss-Legendre kvadratur. Relativ feil er beregnet som |eksakt-numerisk|/eksakt.}
\end{table}

Det er slående at økning i $n$ ikke nødvendigvis medfører høyere nøyaktighet. Tiden går også rett til værs veldig fort,
uten imponerende nøyaktighet i tilnærmingen.

\subsection{Gauss-Laguerre}
Laguerrepolynomene $\ell_k(x)$ er løsningene av differensiallikningen
$$\left(\f{d^2}{dx^2}-\f{d}{dx}+\f{\lambda}{x}-\f{l(l+1}{x^2}\right)f(x)=0$$
med $l\in\mathbb{N}\cup\{0\}$ og $\lambda$ en konstant.
De kan defineres ved Rodrigues' formel:
$$\ell_k(x)=\f{e^x}{k!}\f{d^k}{dx^k}\left(e^{-x}x^k\right),\ \ \ \ k\in \mathbb{N}$$
med $\ell_0(x) =1$ og oppfyller ortogonalitetsrelasjonen
$$\int\limits_0^\infty e^{-x}\ell_i(x)\ell_j(x)dx = \delta_{ij}$$
Disse egner seg dermed godt for integrander $f$ som kan omskrives $f(x) = e^{-x}g(x)$ med integrasjonsgrenser
$0$ og $\infty$ av liknende argumenter som gitt for Gauss-Legendre kvadraturen med indreproduktet
$$\langle s(x)|r(x)\rangle = \int\limits_0^\infty e^{-x}s(x)r(x)dx.$$

Mer generelt egner de seg også for integrander som kan skrives på formen $f(x) = x^\alpha e^{-x}g(x)$.

I vårt tilfelle bytter vi til sfæriske koordinater og kan dermed trekke en faktor $r_i^2e^{-4r_i}$ ut av integranden for $i=1,2$.
Vi substituerer $u_i = 4r_i$ for å få den ønskede formen, hvilket gir oss en normeringsfaktor på $\f{1}{4^5}$.
Vi bruker Gauss-Legendre kvadratur på de angulære integralene og Gauss-Laguerre på de radielle. Dette kan nok en gang
implementeres som en seksdobbel sum.

Tabell 2 viser resultatene for denne metoden.


\begin{table}[h!]
 \centering
 \begin{tabular}{|c|c|c|c|}\hline
 n &Beregning &Relativ feil &Tid [s] \\ \hline
 10      &0.186457      &0.0327256      &0.14\\
 15      &0.189759      &0.0155978      &1.59\\
 20      &0.191082      &0.00873564      &8.77\\
 25      &0.191741      &0.00531718      &33.74\\
 30      &0.192114      &0.00338234      &100.91\\ \hline
  
 \end{tabular}
\caption{Resultater for Gauss-Laguerre kvadratur. Relativ feil er beregnet som |eksakt-numerisk|/eksakt.}
\end{table}

Vi ser at allerede ved $n=15$ er disse beregningene omtrent så nøyaktige som Gauss-Legendre metoden klarer å bli og i
tillegg synker den relative feilen som funksjon av $n$. For $n=20$ er den mer nøyaktig enn Gauss-Laguerre er for noen $n$
og kun på 8.77 sekunder, langt raskere enn forrige metode. Allikevel konvergerer denne tregt. For høyere $n$ vokser
tiden algoritmen bruker raskt og når den bruker 100 sekunder er den fortsatt bare nøyaktig til tredje desimal.

Det er ikke rart at tiden øker raskt med økende $n$ ettersom vi utfører en seksdobbel sum over $n$.

Det er også verdt å merke seg mulige problemer med nevneren i integranden i tilfellet med sfæriske koordinater. Siden
vi må bruke
$$|\vr_1-\vr_2| = \sqrt{r_1^2+r_2^2-2r_1r_2\cos\beta}$$
med $$\cos\beta = \cos\theta_1\cos\theta_2 +\sin\theta_1\sin\theta_2\cos(\phi_1-\phi_2)$$
ettersom det er vanskelig å implementere trikset som brukes i den analytiske utregningen når vi må gjøre alle integralene
på en gang. Når $\cos\beta = 1$ og $r_1$ er lik $r_2$ skal argumentet til kvadratroten bli null. Dessverre jobber vi med
numeriske flyttall, så argumentet kan lett bli et bitte lite negativt tall og utregningen vil gi $nan$. Dette kan løses
ved å innføre en $if$-test som setter integranden til null hver gang argumentet til kvadratroten blir mindre enn
en liten $\epsilon>0$.

\subsection{Monte-Carlo}
Hvis integranden kan faktoriseres $f(x) = p(x)g(x)$ hvor $p(x)$ kan tolkes som en sannsynlighetsfordeling (dvs. er normaliserbar
og positiv mellom integrasjonsgrensene) vil det å finne
integralet av $f(x)$ tilsvare å finne forventningsverdien til $g$ under sannynlighetsfordelingen $p$:
$$I = \int f(x)dx = \int p(x)g(x) dx = \langle g \rangle.$$

Monte-Carlometoden går ut på å trekke $N$ tilfeldige $p$-fordelte tall for å gi et estimat:
$$\langle g\rangle_N = \sum\limits_{i=1}^Ng(x_i)p(x_i)$$
og gjøre dette $M$ ganger for å beregne
$$\langle I\rangle_M = \f{1}{M}\sum\limits_{i=1}^{M}\langle g\rangle_{N,i}$$

som vil konverge mot $I$ når $M\rightarrow \infty$ ved store talls lov.

Med ``Brute force'' Monte-Carlo beholder vi integranden som den er og lar $p(x)=1$ være den uniforme fordelingen
på $[0,1]$. For generelle intervaller gjennomfører vi en substitusjon som før.

I vårt tilfelle kjører vi for kartesiske koordinater. Velger $2.3\approx\infty$ som før og substituerer enhver koordinat
$x_i = -2.3+4.6t_i$ hvor $t_i$ er trukket fra den uniforme fordelingen på $[0,1]$. Dette medfører en jacobideterminant
på $(4.6)^6$. Tabell 3 viser resultatene fra denne metoden.


\begin{table}[h!]
 \centering
 \begin{tabular}{|c|c|c|c|c|}\hline
 n &Beregning &Relativ feil &Standardavvik &Tid [s] \\ \hline
1000      &0.152286      &0.209997             &0.0667105       &0\\
10000      &0.167583      &0.13064             &0.0705365       &0\\
100000      &0.214741      &0.114002             &0.0487582       &0.03\\
1000000      &0.190043      &0.0141255             &0.00971044       &0.21\\
10000000      &0.18422      &0.0443322             &0.00382728       &2.15\\ \hline
  
 \end{tabular}
\caption{Resultater for Brute force Monte-Carlo. Tidsoppløsningen er kun på 0.01s.}
\end{table}

Denne metoden er også ustabil da vi ser at den relative feilen øker fra $n=10^6$ til $n=10^7$. Til gjengjeld er den
veldig rask!


Med ``Importance sampling'' trekker vi ut en sannsynlighetsfordeling fra integranden og gjennomfører en substitusjon
slik at vi fortsatt kan trekke tilfeldige tall i $[0,1]$.

I vårt tilfelle er det mest naturlig å bruke sfæriske koordinater og trekke ut eksponensialfordelingen for de radielle
integralene mens vi holder oss til ``Brute force''-metoden for de angulære integralene.

Vi lar $u_i = 4 r_i$ for å normere eksponensialfunksjonene og substituerer deretter
$$u_i = -\ln(1-x_i) \Leftrightarrow x_i = 1-e^{-u_i}$$
slik at $u_i\in[0,\infty)$ når $x_i\in[0,1)$ og $dx_i = e^{-u_i}du_i$.

Med substitusjonene $\theta_i = 2\pi t_i$ og $\phi_i = \pi s_i$ så $t_i, s_i\in[0,1]$ blir jacobideterminanten
$$\f{\pi^4}{4}e^{-(u1+u2)}.$$

Tabell 4 viser resultatene for denne metoden.


\begin{table}[h!]
 \centering
 \begin{tabular}{|c|c|c|c|c|}\hline
 n &Beregning &Relativ feil &Standardavvik &Tid [s] \\ \hline
1000      &0.15921      &0.174076             &0.0159906       &0\\
10000      &0.195697      &0.0152061             &0.00938424       &0\\
100000      &0.191029      &0.00900808             &0.00312452       &0.03\\
1000000      &0.19335      &0.00303307             &0.00103123       &0.3\\
10000000      &0.192846      &0.000414082             &0.000328002       &2.95\\ \hline
 \end{tabular}
\caption{Resultater for Monte-Carlo med Importance sampling. Tidsoppløsningen er kun på 0.01s.}
\end{table}

Vi ser at resultatene er de mest nøyaktige av alle metodene og i tillegg er denne bare strået tregere enn Brute force
Monte-Carlo, langt raskere enn kvadraturmetodene!

\section{Konklusjon}
Monte-Carlo metodene var langt raskere enn kvadraturmetodene. Dette er utvilsomt fordi man slipper en seksdobbel løkke
i Monte-Carlo. Det fører også til at man kan øke $n$ betraktelig og få veldig nøyaktige resultater. Det hadde vært gøy
å regne ut ``variansen'' til integranden analytisk:
$$\sigma^2 = \int f(\vr_1, \vr_2)^2 d\vr_1d\vr_2 - \left(\int f(\vr_1, \vr_2) d\vr_1d\vr_2\right)^2.$$
Dette for å sammenlikne hvor nøyaktig vi får beregnet standardavviket som er et nyttig mål på feilen vi gjør.
Jeg tror det er mulig og kom ganske langt ved hjelp av meget snedige triks, men det ble dessverre ikke tid til å fullføre.


For den fysiske problemstillingen vi jobber med er tilnærmingen til bølgefunksjonen ikke ideell. Den tar ikke
hensyn til at sannsynligheten burde være null for $r_1=r_2$, men har derimot toppunkt for $r_1=r_2=0$ hvilket ikke
er en lovlig tilstand. Det kunne vært interessant å prøve med bølgefunksjonstilnærminger som tar hensyn til at elektronene
frastøter hverandre og sjekke hvor mye de resultatene avviker fra de vi har funnet her. Det kan også være interessant
å sjekke med eksperimentelle data for å finne ut hvor nær den faktiske verdien vi kom med denne enkle tilnærmingen
til topartikkelbølgefunksjonen.

\end{document}

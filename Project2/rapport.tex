\documentclass[norsk, 12pt]{article}

\usepackage[T1]{fontenc}    % Riktig fontencoding
\usepackage[utf8]{inputenc} % Riktig tegnsett
\usepackage{babel}          % Ordelingsregler, osv
\usepackage{graphicx}       % Inkludere bilder
\usepackage{booktabs}       % Ordentlige tabeller
\usepackage{url}            % Skrive url-er
\usepackage{textcomp}       % Den greske bokstaven micro i text-mode
%\usepackage{units}          % Skrive enheter riktig
\usepackage{float}          % Figurer dukker opp der du ber om
%\usepackage{lipsum}         % Blindtekst
\usepackage{amsmath}        % Mattestæsj
\usepackage{listings}       % Kodetekst
%\usepackage{lipsum} 	    % Package to generate dummy text throughout this template
%\usepackage[lined,boxed,commentsnumbered]{algorithm_2e}
\usepackage{multicol} 	    % Used for the two-column layout of the document
\usepackage[a4paper,margin=1.0in]{geometry}
%\usepackage{comment}
\usepackage{amsfonts}

% Egendefinerte kommandoer
\newcommand{\f}{\frac}
\newcommand{\ddr}{\frac{d}{dr}}

% JF i margen
\makeatletter
\renewcommand{\subsubsection}{\@startsection{subsubsection}{3}{-2cm}%
{-\baselineskip}{0.5\baselineskip}{\bf\large}}
\makeatother
\newcommand{\jf}[1]{\subsubsection*{JF #1}\vspace*{-2\baselineskip}}

% Skru av seksjonsnummerering
\setcounter{secnumdepth}{-1}

%, trim = 1cm 7cm 1cm 7cm % PDF-filer som bilde

\begin{document}

% Forside
\begin{titlepage}
\begin{center}

\textsc{\Large FYS4150 - Computational Physics}\\[0.5cm]
\rule{\linewidth}{0.5mm} \\[0.4cm]
{ \huge \bfseries  PROJECT 2}\\[0.10cm]
\rule{\linewidth}{0.5mm} \\[1.5cm]
\textsc{\Large }\\[1.5cm]
\textsc{}\\[1.5cm]

% Av hvem?
%\begin{minipage}{0.49\textwidth}
 %   \begin{center} \large
  %      Henrik Sverre Limseth\\ \url{henrisli@uio.no} \\[0.8cm]
   % \end{center}
%\end{minipage}
%\bigskip \\
\begin{minipage}{0.49\textwidth}
    \begin{center} \large
        Henrik Sverre Limseth\\ \url{henrisli@uio.no} \\[0.8cm]
    \end{center}
\end{minipage}

%\vfill

% Dato nederst
\large{Dato: \today}

\end{center}
\end{titlepage}

\section{Sammendrag}
I dette prosjektet løser Schrödingerlikningen for to elektroner i et tredimensjonalt harmonisk
oscillatorpotensial, i laveste dreieimpulstilstand ($\ell=0$), med og uten den frastøtende Coulombvekselvirkningen mellom dem. Dette gjøres ved Jacobis 
rotasjonsmetode.

%\begin{multicols*}{2}
\section{Problemet}
Begynner med å se på den radielle Schrödingerlikningen for ett elektron:
\small
$$ -\f{\hbar^2}{2m}\left(\f{1}{r^2}\ddr r^2\ddr -\f{\ell(\ell+1)}{r^2}\right)R + VR=ER$$

I vårt tilfelle er $V(r) = \f{1}{2}m\omega^2r^2$ hvor $\omega$ er svingefrekvensen. For å forenkle
substituerer vi $R(r)= \f{1}{r}u(r)$ og introduserer dimensjonsløs variabel $\rho = \f{r}{\alpha}$ med
$\alpha = \sqrt{\f{\hbar}{m\omega}}$. Deretter definerer vi $\lambda = \f{2E}{\hbar\omega}$ og siden vi
setter $\ell=0$ ender vi etter litt algebra opp med likningen
$$\left(-\f{d^2}{d\rho^2}+\rho^2\right)u(\rho) = \lambda u(\rho).$$
Denne må ha randbetingelser $u(0)=0$ (ellers divergerer $R(0)$) og $\lim\limits_{\rho\rightarrow\infty}u(\rho)=0$ 
(for normalisering). Det er altså denne likningen vi ønsker å løse numerisk i første omgang.

Vi går så videre til to-elektronproblemet med en helt analog tilnærming. 
Den radielle Schrödingerlikningen for $u(r)$ for dette systemet blir:

\begin{align*}
 &\bigg[-\f{\hbar^2}{2m}\left(\f{d^2}{dr_1^2}+\f{d^2}{dr_2^2}\right)+\f{1}{2}m\omega^2\left(r_1^2+r_2^2\right)+ \f{\beta e^2}{|\vec r_1-\vec r_2|}\bigg]u(r_1,r_2) = E^{(2)}u(r_1,r_2)
\end{align*}

hvor $\beta e^2=1.44$eVnm.
For å forenkle innføres nye koordinater: den relative avstanden $\vec r = \vec r_1-\vec r_2$ og massesenterkoordinaten
$\vec R=\f{1}{2}(\vec r_1+\vec r_2)$. Siden løsningene er separable: $u(r,R) = \psi(r)\phi(R)$ og  energien er summen
av massesenterenergien og den relative energien mellom elektronene: $E^{(2)}=E_r+E_R$ ender vi opp med to likninger:
en i massesenterkoordinaten og en i den relative avstanden. Vi konsentrerer oss om sistnevnte, som blir:
$$\left(-\f{\hbar^2}{m}\f{d^2}{dr^2}+\f{1}{4}m\omega^2r^2+\f{\beta e^2}{r}\right)\psi(r)=E_r\psi(r).$$
På samme måte som før introduseres den dimensjonsløse variablen $\rho = \f{r}{\alpha}$, denne gangen med 
$\alpha = \f{\hbar^2}{m\beta e^2}$. For å gjøre likningen penere introduseres også ``frekvensen''
$$\omega_r = \f{m\omega\alpha^2}{2\hbar} = \f{ h^3\epsilon_0^2\omega}{\pi me^4}$$ hvor $h$ er Plancks konstant og $\epsilon_0$
er tomromspermittiviteten, og 
$$\lambda = \f{m\alpha^2}{\hbar^2}E_r = \f{4h^2\epsilon_0^2}{m e^4}E_r.$$
Den radielle Schrödingerlikningen blir til slutt

$$\left(-\f{d^2}{d\rho^2}+\omega_r^2\rho^2+\f{1}{\rho}\right)\psi(\rho) = \lambda\psi(\rho).$$

Vi ser at den er lik likningen for ett elektron hvis vi endrer potensialet $\rho^2\rightarrow \omega_r^2\rho^2+\f{1}{\rho}$.


\section{Metode}
Før diskretisering av likningen kan finne sted må et intervall av interesse defineres, siden vi ikke kan diskretisere
hele $[0,\infty)$. Fordi potensialet i første tilnærming (ett elektron) $V\propto r^2$ raskt øker for store $r$ (store $\rho$) og har bunnpunkt i $r=0$ er
det rimelig å anta at bølgefunksjonen er maksimal i et intervall $\rho \in [0, \rho_{max}]$ og raskt dør ut for 
store $\rho$.

For to vekselvirkende elektroner har vi $V(\rho) = \omega_r^2\rho^2+\f{1}{\rho}$. For å finne de interessante verdiene
for $\rho$ finner vi det reelle ekstremalpunktet. Dette vil være et bunnpunkt i potensialet siden 
$\lim\limits_{\rho\rightarrow\infty}V(\rho) = \lim\limits_{\rho\rightarrow0} V(\rho) = \infty$ og $V(\rho)$ er kontinuerlig.
Derfor er det naturlig å anta at bølgefunksjonen vil være maksimal i dette området.

\begin{align*}
&\f{dV}{d\rho}=2\omega_r^2\rho-\f{1}{\rho^2}=0\ \ \ \Rightarrow\ \ \ \rho_{0}=\f{1}{\sqrt[3]{2\omega_r^2}}
\end{align*}
For $\omega_r = 0.01$ får vi $\rho_0 \approx 3.68$. Vi trenger altså ikke å velge $\rho_{max}$ veldig stor for å få med
de viktigste delene av bølgefunksjonen.


Deretter bruker vi standard tilnærming til andrederivert:
$$u''(\rho)\approx \f{u(\rho+h)-2u(\rho)+u(\rho-h)}{h^2}$$
hvor $h=\f{\rho_{max}}{n}$ med $n$ lik antall punkter vi har inndelt intervallet i. Diskretiskert blir da $\rho_i = ih$, $u_i = u(\rho_i)$ og $V_i = V(\rho_i)$
for $i=0,1,2,\dots ,n$. Den diskretiserte likningen blir
$$-\f{1}{h^2}(u_{i+1} -2u_i+u_{i-1}) + V_iu_i = \lambda u_i.$$
Vi definerer $d_i = \f{2}{h^2}+V_i^2$ og $e = -\f{1}{h^2}$ og ender opp med
$$eu_{i-1}+d_iu_i+e_{i+1}u_{i+1} = \lambda u_i.$$

Dette kan omskrives som en matriselikning:
$$A \vec u = \lambda \vec u$$
med

$$ A = \left(\begin{matrix}
 d_1 & e & 0 & 0 & \dots & 0 \\
 e & d_2 & e & 0 & \dots & 0 \\
 0 & \ddots & \ddots & \ddots &  & \vdots \\
 \vdots & & \ddots & \ddots & \ddots & 0 \\
 0 & \dots & 0 & e & d_{n-2} & e\\
 0 & \dots & \dots & 0 & e & d_{n-1} \\
\end{matrix}\right)$$ 
og 

$$ \vec u  = \left(\begin{matrix} u_1 \\ u_2 \\ \vdots \\u_{n-1} \end{matrix}\right).$$
Dette er altså et egenverdiproblem og vi kan anvende Jacobis snurretriks.
Denne metoden baserer seg på å utføre similærtransformasjoner på $A$: $A \rightarrow S^{-1}AS$ med $S$ 
$(n-1)\times(n-1)$-rotasjonsmatriser til den roterte matrisen er tilnærmet diagonal, dvs. til alle elementer som 
ikke ligger på diagonalen er tilnærmet lik 0. Da vil diagonalelementene tilnærmet være egenverdiene til $A$.

I tillegg utfører vi alle transformasjonene på en identitetsmatrise $R$: $R \rightarrow S^{-1}RS$. Kolonnene til $R$ vil da
tilnærmet være egenvektorene til $A$, tilsvarende $\vec u$ for forskjellige verdier av $\lambda$, hvilket igjen 
tilsvarer den numeriske løsningen for $u(\rho)$.

Trikset går ut på å for hver iterasjon velge rotasjonsmatrisen $S$ slik at de største ikke-diagonale elementene til $A$
vil være 0 etter transformasjonen. Vi definerer derfor $S$ på følgende måte:

$$s_{kk} = s_{ll} = cos\theta$$
$$\ s_{kl} = s_{lk} = -sin\theta$$
$$s_{ii} = 1,\ \ \ i\neq k,l$$
hvor $k,\ l$ er valgt slik at $a_{kl}=a_{lk}$ ($A$ er symmetrisk) er det største ikke-diagonale elementet i $A$.
Dersom $B = S^{-1}AS$ får vi (med forkortelsen $c=\cos\theta,\ s=\sin\theta$):
$$b_{ii} = a_{ii},\ \ i\neq k,l$$
$$b_{ik} = a_{ik}c - a_{il}s,\ \ i\neq k,l$$
$$b_{il} = a_{il}c + a_{ik}s,\ \ i\neq k,l$$
$$b_{kk} = a_{kk}c^2-2a_{kl}cs+a_{ll}s^2$$
$$b_{ll} = a_{ll} c^2 +2a_{kl}cs + a_{kk}s^2$$
$$b_{kl} = (a_{kk}-a_{ll})cs+a_{kl}(c^2-s^2)$$
hvor vi  vil ha $b_{kl}=0$. Dette er automatisk oppfylt hvis $\theta$ velges slik at
$$\cot2\theta = \f{a_{ll}-a_{kk}}{a_{kl}}\equiv \tau.$$
Definer $t = \tan\theta$. Ved identiteten $\cot2\theta = \f{1}{2}(\cot\theta-\tan\theta)$ oppnår vi løsningene
$$t = -\tau\pm\sqrt{1+\tau^2}.$$


Det er altså to mulige verdier av $t$. Vi trenger å velge den som gjør at $|\theta|\leq\f{\pi}{4}$. Dette er nødvendig
for å sikre konvergens [1]. For å bestemme hvilken $t$ som gjør jobben antar vi $|\theta|\leq\f{\pi}{4}$ og ser hvilke
begrensninger det setter på $t$:

Siden $|\theta|\leq \f{\pi}{4} \Leftrightarrow \cos\theta \geq \f{1}{\sqrt{2}}$ (når $\theta\in [0,2\pi]$) får vi
$$\f{1}{\sqrt{1+t^2}} \geq \f{1}{\sqrt{2}}\  \Leftrightarrow\  2\geq1+t^2$$

$$\Leftrightarrow \ \ t^2\leq 1\ \  \Leftrightarrow\ \ \ |t|\leq1.$$

Vi må altså ha $|\tau\mp\sqrt{1+\tau^2}|\leq1$. Denne ulikheten er tilfredsstilt ved å velge ``$-$'' for $\tau>0$ og ``$+$'' for $\tau<0$.
Dette tilsvarer
$$t = -\tau + \sqrt{1+\tau^2},\ \ \ \text{hvis}\ \ \tau>0$$
$$t = -\tau - \sqrt{1+\tau^2},\ \ \ \text{hvis}\ \ \tau<0$$

For $\tau=0$ er det likegyldig og
$|\theta|= \f{\pi}{4}$. Ved å velge den minste mulige verdien for $t$ sikrer vi dermed $|\theta|\leq\f{\pi}{4}$.

Siden dette valget av $t$ sikrer at $c\in\left[\f{1}{\sqrt{2}}, 1\right]$ medfører det at $(1-c)$ og $\f{1}{c^2}$ vil
anta minimale verdier. Altså vil ethvert ledd i 
$$\|B-A\|_F^2=4(1-c)\sum\limits_{i=1,i\neq k,l}^{n-1}(a_{ik}^2+a_{il}^2)+\f{2a_{kl}^2}{c^2}$$
være minimalt, så Fröbeniusdifferansen til $A$ og den transformerte matrisen $B$ vil være minimal.

Når $t$ er bestemt kan $c$ og $s$ beregnes: $$c = \f{1}{\sqrt{1+t^2}},\ \ s = ct$$
og vi kan regne ut de nye matriseelementene. Dette gjentas til det maksimale ikke-diagonale elementet i matrisen
er mindre enn en toleranse $\epsilon\approx0$ (i dette prosjektet er $\epsilon=10^{-8}$ brukt).

Algoritmen blir som følger:

\textbf{1)} Finn ikke-diagonalt element med høyest verdi: $$|a_{kl}|=\max\limits_{i\neq j}|a_{ij}|$$ og plasseringen $k,l$ til
dette elementet.

\textbf{2)} Regn ut $$\tau=\f{a_{ll}-a_{kk}}{2a_{kl}}$$ og bestem $$t=-\tau\pm\sqrt{1+\tau^2}$$ med ``$+$'' for $\tau\geq0$ og ``$-$'' for 
$\tau<0$. Regn deretter ut $$c = \f{1}{\sqrt{1+t^2}}\ \  \text{og}\ \ \  s=ct.$$

\textbf{3)} Regn ut rotasjonsmatriseelementene: $$s_{ii} = 1,\ \ i\neq k,l$$
$$s_{ll} = s_{kk} = c$$
$$s_{lk}=-s_{kl} = s$$

\textbf{4)} Regn ut elementene $b_{ij}$ til $B = S^{T}AS$ og redefiner matrisen $A$ til å ha disse elementene.

\textbf{5)} Oppdater matrisen som har egenvektorene til $A$ som kolonner: $$R\rightarrow S^TRS$$

\textbf{6)} Gjenta \textbf{1} - \textbf{5} til $\max\limits_{i\neq j}\{a_{ij}^2\}<\epsilon$.

\section{Resultater}
Vi begynner med å løse likningen for ett elektron for forskjellige verdier av $\rho_{max}$, med $n=100$. Vi kjenner
de tre laveste egenverdiene i dette tilfellet: $\lambda_0 = 3,\ \ \lambda_1 = 7,\ \ \lambda_2 = 11$. Dermed kan vi sjekke
for hvilken verdi av $\rho_{max}$ vi får de mest nøyaktige resultatene. Tabell 1 viser de beregnede verdiene for
$\lambda_0$, $\lambda_1$ og $\lambda_2$ for forskjellige $\rho_{max}$ med $n=100$, kolonnen til høyre viser antall
iterasjoner algoritmen brukte på å få alle ikke-diagonale elementer mindre enn $\epsilon = 10^{-8}$.
\begin{table}[h!]
\centering
 \begin{tabular}{|c|c|c|c|c|}\hline
  $\rho_{max}$ & $\lambda_0$ & $\lambda_1$ & $\lambda_2$ & Antall iterasjoner\\ \hline
  1 & 10.1504 & 39.7864 & 89.0886 & 17114\\
  2 & 3.5294 & 11.166 & 23.5153 & 16755\\
  3 & 3.01188 & 7.32662 & 12.939 & 16588\\
  4 & 2.99953 & 7.00088 & 11.0724 & 16405\\
  4.5 & 2.99937 & 6.99696 & 10.998 & 16270\\
  5 & 2.99922 & 6.99609 & 10.9907 & 16217\\
  6 & 2.99887 & 6.99234 & 10.9836 & 16021\\
  7 & 2.99847 & 6.99234 & 10.9813 & 15835\\
  8 & 2.998 & 6.98999 & 10.9755 & 15696\\ \hline
 \end{tabular}
 \caption{De tre minste egenverdiene beregnet for forskjellige $\rho_{max}$ ved $n=100$}
 \end{table}
 
 
For lave $\rho_{max}$ er resultatene helt ubrukelige. Dette er antakeligvis fordi vi ikke får med oss deler av rommet
der potensialet fortsatt er relativt lite og bølgefunksjonen har gode levekår. For høye $\rho_{max}$ blir også resultatene
mindre nøyaktige. Intuitivt skulle man tenke at å ta med større deler av rommet gir mer nøyaktige beregninger, men antakeligvis
trenger man høyere $n$ for å realisere dette.

Rundt $\rho_{max}$ lik 3-4.5 slutter metoden å tilnærme egenverdiene
nedenfra og begynner å tilnærme ovenfra. Det tyder på at det er i dette området vi får mest nøyaktige beregninger.
 Siden de høyere egenverdiene vil konvergere
tregest er det lurt å velge den $\rho_{max}$-verdien som gir mest nøyaktig $\lambda_2$.
$\rho_{max} = 4.5$ blir valgt for å studere ett-elektronsystemet videre, ettersom denne verdien gir rimelig nøyaktige 
$\lambda_0$ og $\lambda_1$ og viktigst: den mest nøyaktige $\lambda_2$.

Vi legger merke til at antall iterasjoner minker langsomt med voksende $\rho_{max}$ for konstant $n$.

Tabell 2 viser de tre minste egenverdiene beregnet for forskjellige $n$ med $\rho_{max}=4.5$. 
\begin{table}[h!]
 \centering

\begin{tabular}{|c|c|c|c|}\hline
 $n$ & $\lambda_0$ & $\lambda_1$ & $\lambda_2$\\ \hline
 100 & 2.99937 & 6.99696 & 10.998\\
 200 & 2.99984 & 6.99933 & 11.0038 \\
 300 & 2.99978 & 6.99978 & 11.0049 \\
 400 & 2.99996 & 6.99993 & 11.0053 \\ \hline 
\end{tabular}
\caption{Beregnede egenverdier for $\rho_{max} = 4.5$}
\end{table}

Det ser ut til at $\lambda_2$ begynner å divergere for høye $n$. Det virker som 4.5 var en for lav verdi for $\rho_{max}$, så vi
prøver på nytt med $\rho_{max} = 5$, se Tabell 3.

\begin{table}[h!]
 \centering
 \begin{tabular}{|c|c|c|c|c|c|}\hline
  $n$ & $\lambda_0$ & $\lambda_1$ & $\lambda_2$ & Tid [s] & Antall iterasjoner \\ \hline
  100 & 2.99923 & 6.99617 & 10.9908 & 0.19 & 16475\\
  200 & 2.99981 & 6.99904 & 10.9978 & 2.94 & 66828 \\
  300 & 2.99991 & 6.99957 & 10.9991 & 14.4 & 150798 \\
  400 & 2.99995 & 6.99976 & 10.9996 & 44.66 & 269021 \\
  500 & 2.99997 & 6.99985 & 10.9998 & 108.05 & 422320 \\ \hline 
 \end{tabular}
\caption{Beregnede egenverdier for $\rho_{max} = 5$, sammen med tiden algoritmen brukte på å kjøre og antall iterasjoner
den brukte.}
\end{table}

Vi ser at $\lambda_0$ og $\lambda_1$ konvergerer omtrent like raskt for $\rho_{max}=5$ som for $\rho_{max}=4.5$, men
for $\rho_{max}=5$ divergerer ikke $\lambda_2$. For $n=500$ er resultatene veldig nøyaktige og kun med en liten feil i
fjerde desimal for $\lambda_1$ og $\lambda_2$. Vi legger også merke til at antall iterasjoner vokser kraftig med $n$.

\begin{figure}
 \centering
 \includegraphics[scale=0.7]{transformasjoner.png}
 \caption{Antall similærtransformasjoner som trengs før alle ikke-diagonale elementer omtrent er null som funksjon av
 $\rho_{max}$ og $n$.}
\end{figure}

Figur 1 viser antall similærtransformasjoner algoritmen bruker som funksjon av $\rho_{max}$ og $n$. Som funksjon av $\rho_max$
synker det langsomt med økende $\rho_{max}$, det kan se ut som et lineært forhold. Som funksjon av $n$ vokser det 
raskere og raskere for høyere $n$, forholdet ser eksponensielt ut her, eller i det minste kvadratisk. Dessverre er det altfor få datapunkter til å kunne
fastslå noen streng sammenheng.

For å undersøke dette nærmere plottes tiden algoritmen brukte mot antall punkter og antall tranformasjoner mot tiden
(Figur 2). Det ser ut til at antall transformasjoner er proporsjonal med logaritmen til tiden algoritmen bruker, mens
tiden algoritmen bruker ser veldig ut til å vokse eksponensielt med antall steg. Det er veldig rart at antall
transformasjoner og tiden algoritmen bruker ikke henger sammen lineært! Ut ifra plottene virker det i hvert fall
rimelig å anta
$$t\propto e^{n^\alpha}\Rightarrow \text{Transformasjoner} \propto \ln e^{n^\alpha} = n^\alpha $$
for en $\alpha>0$.

\begin{figure}
 \centering
 \includegraphics[scale=0.7]{tidsomfunkavn.png}
 \caption{Tid algoritmen bruker mot dimensjonen av matrisen og antall transformasjoner som må gjennomføres mot tiden}
\end{figure}


%\end{multicols*}

\section{Kilder}
1: http://web.stanford.edu/class/cme335/lecture7.pdf



\end{document}

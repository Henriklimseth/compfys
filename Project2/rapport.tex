\documentclass[norsk, 12pt]{article}

\usepackage[T1]{fontenc}    % Riktig fontencoding
\usepackage[utf8]{inputenc} % Riktig tegnsett
\usepackage{babel}          % Ordelingsregler, osv
\usepackage{graphicx}       % Inkludere bilder
\usepackage{booktabs}       % Ordentlige tabeller
\usepackage{url}            % Skrive url-er
\usepackage{textcomp}       % Den greske bokstaven micro i text-mode
%\usepackage{units}          % Skrive enheter riktig
\usepackage{float}          % Figurer dukker opp der du ber om
%\usepackage{lipsum}         % Blindtekst
\usepackage{amsmath}        % Mattestæsj
\usepackage{listings}       % Kodetekst
%\usepackage{lipsum} 	    % Package to generate dummy text throughout this template
%\usepackage[lined,boxed,commentsnumbered]{algorithm_2e}
\usepackage{multicol} 	    % Used for the two-column layout of the document
\usepackage[a4paper,margin=1.0in]{geometry}
%\usepackage{comment}
\usepackage{amsfonts}
\usepackage{parskip}

% Egendefinerte kommandoer
\newcommand{\f}{\frac}
\newcommand{\ddr}{\frac{d}{dr}}

% JF i margen
\makeatletter
\renewcommand{\subsubsection}{\@startsection{subsubsection}{3}{-2cm}%
{-\baselineskip}{0.5\baselineskip}{\bf\large}}
\makeatother
\newcommand{\jf}[1]{\subsubsection*{JF #1}\vspace*{-2\baselineskip}}

% Skru av seksjonsnummerering
\setcounter{secnumdepth}{-1}

%, trim = 1cm 7cm 1cm 7cm % PDF-filer som bilde

\begin{document}

% Forside
\begin{titlepage}
\begin{center}

\textsc{\Large FYS4150 - Computational Physics}\\[0.5cm]
\rule{\linewidth}{0.5mm} \\[0.4cm]
{ \huge \bfseries  PROJECT 2}\\[0.10cm]
\rule{\linewidth}{0.5mm} \\[1.5cm]
\textsc{\Large }\\[1.5cm]
\textsc{}\\[1.5cm]

% Av hvem?
%\begin{minipage}{0.49\textwidth}
 %   \begin{center} \large
  %      Henrik Sverre Limseth\\ \url{henrisli@uio.no} \\[0.8cm]
   % \end{center}
%\end{minipage}
%\bigskip \\
\begin{minipage}{0.49\textwidth}
    \begin{center} \large
        Henrik Sverre Limseth\\ \url{henrisli@uio.no} \\[0.8cm]
    \end{center}
\end{minipage}

%\vfill

% Dato nederst
\large{Dato: \today}

\end{center}
\end{titlepage}

\section{Sammendrag}
I dette prosjektet løses Schrödingerlikningen for ett elektron i et tredimensjonalt harmonisk
oscillatorpotensial, i laveste dreieimpulstilstand ($\ell=0$). Dette utvides så til tilfellet to elektroner med og uten den
frastøtende Coulombvekselvirkningen mellom dem. Likningen løses numerisk med Jacobis rotasjonsmetode og Armadillos 
egenverdi/egenvektorfunksjoner. Resultatene sammenliknes med analytiske løsninger der disse finnes.

%\begin{multicols*}{2}
\section{Problemet}
Begynner med å se på den radielle Schrödingerlikningen for ett elektron:
\small
$$ -\f{\hbar^2}{2m}\left(\f{1}{r^2}\ddr r^2\ddr -\f{\ell(\ell+1)}{r^2}\right)R + VR=ER$$

I vårt tilfelle er $V(r) = \f{1}{2}m\omega^2r^2$ hvor $\omega$ er svingefrekvensen. For å forenkle
substituerer vi $R(r)= \f{1}{r}u(r)$ og introduserer dimensjonsløs variabel $\rho = \f{r}{\alpha}$ med
$\alpha = \sqrt{\f{\hbar}{m\omega}}$. Deretter definerer vi $\lambda = \f{2E}{\hbar\omega} = 2\cdot(2n+\ell+\f{3}{2})$ og siden vi
setter $\ell=0$ ender vi etter litt algebra opp med likningen
$$\left(-\f{d^2}{d\rho^2}+\rho^2\right)u(\rho) = \lambda u(\rho).$$
Denne må ha randbetingelser $u(0)=0$ (ellers divergerer $R(0)$) og $\lim\limits_{\rho\rightarrow\infty}u(\rho)=0$ 
(for normalisering). Det er altså denne likningen vi ønsker å løse numerisk i første omgang.

Vi går så videre til to-elektronproblemet med en helt analog tilnærming. 
Den radielle Schrödingerlikningen for $u(r)$ for dette systemet blir:

\begin{align*}
 &\bigg[-\f{\hbar^2}{2m}\left(\f{d^2}{dr_1^2}+\f{d^2}{dr_2^2}\right)+\f{1}{2}m\omega^2\left(r_1^2+r_2^2\right)+ \f{\beta e^2}{|\vec r_1-\vec r_2|}\bigg]u(r_1,r_2) = E^{(2)}u(r_1,r_2)
\end{align*}

hvor $\beta e^2=1.44$eVnm.
For å forenkle innføres nye koordinater: den relative avstanden $\vec r = \vec r_1-\vec r_2$ og massesenterkoordinaten
$\vec R=\f{1}{2}(\vec r_1+\vec r_2)$. Siden løsningene er separable: $u(r,R) = \psi(r)\phi(R)$ og  energien er summen
av massesenterenergien og den relative energien mellom elektronene: $E^{(2)}=E_r+E_R$ ender vi opp med to likninger:
en i massesenterkoordinaten og en i den relative avstanden. Vi konsentrerer oss om sistnevnte, som blir:
$$\left(-\f{\hbar^2}{m}\f{d^2}{dr^2}+\f{1}{4}m\omega^2r^2+\f{\beta e^2}{r}\right)\psi(r)=E_r\psi(r).$$
På samme måte som før introduseres den dimensjonsløse variablen $\rho = \f{r}{\alpha}$, denne gangen med 
$\alpha = \f{\hbar^2}{m\beta e^2}$. For å gjøre likningen penere introduseres også ``frekvensen''
$$\omega_r = \f{m\omega\alpha^2}{2\hbar} = \f{ h^3\epsilon_0^2\omega}{\pi me^4}$$ hvor $h$ er Plancks konstant og $\epsilon_0$
er tomromspermittiviteten, og 
$$\lambda = \f{m\alpha^2}{\hbar^2}E_r = \f{4h^2\epsilon_0^2}{m e^4}E_r.$$
Den radielle Schrödingerlikningen blir til slutt

$$\left(-\f{d^2}{d\rho^2}+\omega_r^2\rho^2+\f{1}{\rho}\right)\psi(\rho) = \lambda\psi(\rho).$$

Vi ser at den er lik likningen for ett elektron hvis vi endrer potensialet $\rho^2\rightarrow \omega_r^2\rho^2+\f{1}{\rho}$.


\section{Metode}
Før diskretisering av likningen kan finne sted må et intervall av interesse defineres, siden vi ikke kan diskretisere
hele $[0,\infty)$. Fordi potensialet i første tilnærming (ett elektron) $V\propto r^2$ raskt øker for store $r$ (store $\rho$) og har bunnpunkt i $r=0$ er
det rimelig å anta at bølgefunksjonen er maksimal i et intervall $\rho \in [0, \rho_{max}]$ og raskt dør ut for 
store $\rho$.

For to vekselvirkende elektroner har vi $V(\rho) = \omega_r^2\rho^2+\f{1}{\rho}$. For å finne de interessante verdiene
for $\rho$ finner vi det reelle ekstremalpunktet. Dette vil være et bunnpunkt i potensialet siden 
$\lim\limits_{\rho\rightarrow\infty}V(\rho) = \lim\limits_{\rho\rightarrow0} V(\rho) = \infty$ og $V(\rho)$ er kontinuerlig.
Derfor er det naturlig å anta at bølgefunksjonen vil være maksimal i dette området.

\begin{align*}
&\f{dV}{d\rho}=2\omega_r^2\rho-\f{1}{\rho^2}=0\ \ \ \Rightarrow\ \ \ \rho_{0}=\f{1}{\sqrt[3]{2\omega_r^2}}
\end{align*}
For $\omega_r = 0.01$ får vi $\rho_0 \approx 17.1$. I tillegg vil ser vi at $V$ vil stige langsomt for store $\rho$,
men divergere for $\rho$ nær 0. Det er derfor naturlig å studere situasjonen for større og større $\rho$ jo lavere
$\omega_r$ er.
En blanding av denne teorien og eksperimentell tilnærming til de eksakte resultatene vi er oppgitt for
$\omega_r = 0.25$ og $\omega_r=0.05$ ga denne formelen:
$$\rho_{max} = \f{1}{\sqrt[3]{2\omega_r^2}} +\f{4}{5\omega_r} + 5$$
hvor det første leddet passer på at $\rho_{max}$ er større enn $\rho_0$, det andre tar hensyn til at potensialet
stiger tregere for lavere $\omega_r$ og det tredje sørger for at vi får med oss en stor nok del av rommet selv for høye
$\omega_r$.


Deretter bruker vi standard tilnærming til andrederivert:
$$u''(\rho)\approx \f{u(\rho+h)-2u(\rho)+u(\rho-h)}{h^2}$$
hvor $h=\f{\rho_{max}}{n}$ med $n$ lik antall punkter vi har inndelt intervallet i. Diskretiskert blir da $\rho_i = ih$, $u_i = u(\rho_i)$ og $V_i = V(\rho_i)$
for $i=0,1,2,\dots ,n$. Den diskretiserte likningen blir
$$-\f{1}{h^2}(u_{i+1} -2u_i+u_{i-1}) + V_iu_i = \lambda u_i.$$
Vi definerer $d_i = \f{2}{h^2}+V_i^2$ og $e = -\f{1}{h^2}$ og ender opp med
$$eu_{i-1}+d_iu_i+e_{i+1}u_{i+1} = \lambda u_i.$$

Dette kan omskrives som en matriselikning:
$$A \vec u = \lambda \vec u$$
med

$$ A = \left(\begin{matrix}
 d_1 & e & 0 & 0 & \dots & 0 \\
 e & d_2 & e & 0 & \dots & 0 \\
 0 & \ddots & \ddots & \ddots &  & \vdots \\
 \vdots & & \ddots & \ddots & \ddots & 0 \\
 0 & \dots & 0 & e & d_{n-2} & e\\
 0 & \dots & \dots & 0 & e & d_{n-1} \\
\end{matrix}\right)$$ 
og 

$$ \vec u  = \left(\begin{matrix} u_1 \\ u_2 \\ \vdots \\u_{n-1} \end{matrix}\right).$$
Dette er altså et egenverdiproblem og vi kan anvende Jacobis snurretriks.
Denne metoden baserer seg på å utføre similærtransformasjoner på $A$: $A \rightarrow S^{-1}AS$ med $S$ 
$(n-1)\times(n-1)$-rotasjonsmatriser til den roterte matrisen er tilnærmet diagonal, dvs. til alle elementer som 
ikke ligger på diagonalen er tilnærmet lik 0. Da vil diagonalelementene tilnærmet være egenverdiene til $A$.

I tillegg utfører vi alle transformasjonene på en identitetsmatrise $R$: $R \rightarrow S^{-1}RS$. Kolonnene til $R$ vil da
tilnærmet være egenvektorene til $A$, tilsvarende $\vec u$ for forskjellige verdier av $\lambda$, hvilket igjen 
tilsvarer den numeriske løsningen for $u(\rho)$.

Trikset går ut på å for hver iterasjon velge rotasjonsmatrisen $S$ slik at de største ikke-diagonale elementene til $A$
vil være 0 etter transformasjonen. Vi definerer derfor $S$ på følgende måte:

$$s_{kk} = s_{ll} = cos\theta$$
$$\ s_{kl} = s_{lk} = -sin\theta$$
$$s_{ii} = 1,\ \ \ i\neq k,l$$
hvor $k,\ l$ er valgt slik at $a_{kl}=a_{lk}$ ($A$ er symmetrisk) er det største ikke-diagonale elementet i $A$.
Dersom $B = S^{-1}AS$ får vi (med forkortelsen $c=\cos\theta,\ s=\sin\theta$):
$$b_{ii} = a_{ii},\ \ i\neq k,l$$
$$b_{ik} = a_{ik}c - a_{il}s,\ \ i\neq k,l$$
$$b_{il} = a_{il}c + a_{ik}s,\ \ i\neq k,l$$
$$b_{kk} = a_{kk}c^2-2a_{kl}cs+a_{ll}s^2$$
$$b_{ll} = a_{ll} c^2 +2a_{kl}cs + a_{kk}s^2$$
$$b_{kl} = (a_{kk}-a_{ll})cs+a_{kl}(c^2-s^2)$$
hvor vi  vil ha $b_{kl}=0$. Dette er automatisk oppfylt hvis $\theta$ velges slik at
$$\cot2\theta = \f{a_{ll}-a_{kk}}{a_{kl}}\equiv \tau.$$
Definer $t = \tan\theta$. Ved identiteten $\cot2\theta = \f{1}{2}(\cot\theta-\tan\theta)$ oppnår vi løsningene
$$t = -\tau\pm\sqrt{1+\tau^2}.$$


Det er altså to mulige verdier av $t$. Vi trenger å velge den som gjør at $|\theta|\leq\f{\pi}{4}$. Dette er nødvendig
for å sikre konvergens [1]. For å bestemme hvilken $t$ som gjør jobben antar vi $|\theta|\leq\f{\pi}{4}$ og ser hvilke
begrensninger det setter på $t$:

Siden $|\theta|\leq \f{\pi}{4} \Leftrightarrow \cos\theta \geq \f{1}{\sqrt{2}}$ (når $\theta\in [0,2\pi]$) får vi
$$\f{1}{\sqrt{1+t^2}} \geq \f{1}{\sqrt{2}}\  \Leftrightarrow\  2\geq1+t^2$$

$$\Leftrightarrow \ \ t^2\leq 1\ \  \Leftrightarrow\ \ \ |t|\leq1.$$

Vi må altså ha $|\tau\mp\sqrt{1+\tau^2}|\leq1$. Denne ulikheten er tilfredsstilt ved å velge ``$-$'' for $\tau>0$ og ``$+$'' for $\tau<0$.
Dette tilsvarer
$$t = -\tau + \sqrt{1+\tau^2},\ \ \ \text{hvis}\ \ \tau>0$$
$$t = -\tau - \sqrt{1+\tau^2},\ \ \ \text{hvis}\ \ \tau<0$$

For $\tau=0$ er det likegyldig og
$|\theta|= \f{\pi}{4}$. Ved å velge den minste mulige verdien for $t$ sikrer vi dermed $|\theta|\leq\f{\pi}{4}$.

Siden dette valget av $t$ sikrer at $c\in\left[\f{1}{\sqrt{2}}, 1\right]$ medfører det at $(1-c)$ og $\f{1}{c^2}$ vil
anta minimale verdier. Altså vil ethvert ledd i 
$$\|B-A\|_F^2=4(1-c)\sum\limits_{i=1,i\neq k,l}^{n-1}(a_{ik}^2+a_{il}^2)+\f{2a_{kl}^2}{c^2}$$
være minimalt, så Fröbeniusdifferansen til $A$ og den transformerte matrisen $B$ vil være minimal.

Når $t$ er bestemt kan $c$ og $s$ beregnes: $$c = \f{1}{\sqrt{1+t^2}},\ \ s = ct$$
og vi kan regne ut de nye matriseelementene. Dette gjentas til det maksimale ikke-diagonale elementet i matrisen
er mindre enn en toleranse $\epsilon\approx0$ (i dette prosjektet er $\epsilon=10^{-8}$ brukt).

Algoritmen blir som følger:

\textbf{1)} Finn ikke-diagonalt element med høyest verdi: $$|a_{kl}|=\max\limits_{i\neq j}|a_{ij}|$$ og plasseringen $k,l$ til
dette elementet.

\textbf{2)} Regn ut $$\tau=\f{a_{ll}-a_{kk}}{2a_{kl}}$$ og bestem $$t=-\tau\pm\sqrt{1+\tau^2}$$ med ``$+$'' for $\tau\geq0$ og ``$-$'' for 
$\tau<0$. Regn deretter ut $$c = \f{1}{\sqrt{1+t^2}}\ \  \text{og}\ \ \  s=ct.$$

\textbf{3)} Regn ut rotasjonsmatriseelementene: $$s_{ii} = 1,\ \ i\neq k,l$$
$$s_{ll} = s_{kk} = c$$
$$s_{lk}=-s_{kl} = s$$

\textbf{4)} Regn ut elementene $b_{ij}$ til $B = S^{T}AS$ og redefiner matrisen $A$ til å ha disse elementene.

\textbf{5)} Oppdater matrisen som har egenvektorene til $A$ som kolonner: $$R\rightarrow S^TRS$$

\textbf{6)} Gjenta \textbf{1} - \textbf{5} til $\max\limits_{i\neq j}\{a_{ij}^2\}<\epsilon$.

\section{Resultater}
Vi begynner med å løse likningen for ett elektron for forskjellige verdier av $\rho_{max}$, med $n=100$. Vi kjenner
de tre laveste egenverdiene i dette tilfellet: $\lambda_0 = 3,\ \ \lambda_1 = 7,\ \ \lambda_2 = 11$. Dermed kan vi sjekke
for hvilken verdi av $\rho_{max}$ vi får de mest nøyaktige resultatene. Tabell 1 viser de beregnede verdiene for
$\lambda_0$, $\lambda_1$ og $\lambda_2$ for forskjellige $\rho_{max}$ med $n=100$, kolonnen til høyre viser antall
iterasjoner algoritmen brukte på å få alle ikke-diagonale elementer mindre enn $\epsilon = 10^{-8}$.
\begin{table}[h!]
\centering
 \begin{tabular}{|c|c|c|c|c|}\hline
  $\rho_{max}$ & $\lambda_0$ & $\lambda_1$ & $\lambda_2$ & Antall iterasjoner\\ \hline
  1 & 10.1504 & 39.7864 & 89.0886 & 17114\\
  2 & 3.5294 & 11.166 & 23.5153 & 16755\\
  3 & 3.01188 & 7.32662 & 12.939 & 16588\\
  4 & 2.99953 & 7.00088 & 11.0724 & 16405\\
  4.5 & 2.99937 & 6.99696 & 10.998 & 16270\\
  5 & 2.99922 & 6.99609 & 10.9907 & 16217\\
  6 & 2.99887 & 6.99234 & 10.9836 & 16021\\
  7 & 2.99847 & 6.99234 & 10.9813 & 15835\\
  8 & 2.998 & 6.98999 & 10.9755 & 15696\\ \hline
 \end{tabular}
 \caption{De tre minste egenverdiene beregnet for forskjellige $\rho_{max}$ ved $n=100$}
 \end{table}
 
 
For lave $\rho_{max}$ er resultatene helt ubrukelige. Dette er antakeligvis fordi vi ikke får med oss deler av rommet
der potensialet fortsatt er relativt lite og bølgefunksjonen har gode levekår. For høye $\rho_{max}$ blir også resultatene
mindre nøyaktige. Intuitivt skulle man tenke at å ta med større deler av rommet gir mer nøyaktige beregninger, men antakeligvis
trenger man høyere $n$ for å realisere dette.

Rundt $\rho_{max}$ lik 3-4.5 slutter metoden å tilnærme egenverdiene
nedenfra og begynner å tilnærme ovenfra. Det tyder på at det er i dette området vi får mest nøyaktige beregninger.
 Siden de høyere egenverdiene vil konvergere
tregest er det lurt å velge den $\rho_{max}$-verdien som gir mest nøyaktig $\lambda_2$.
$\rho_{max} = 4.5$ blir valgt for å studere ett-elektronsystemet videre, ettersom denne verdien gir rimelig nøyaktige 
$\lambda_0$ og $\lambda_1$ og viktigst: den mest nøyaktige $\lambda_2$.

Vi legger merke til at antall iterasjoner minker langsomt med voksende $\rho_{max}$ for konstant $n$.

Tabell 2 viser de tre minste egenverdiene beregnet for forskjellige $n$ med $\rho_{max}=4.5$. 
\begin{table}[h!]
 \centering

\begin{tabular}{|c|c|c|c|}\hline
 $n$ & $\lambda_0$ & $\lambda_1$ & $\lambda_2$\\ \hline
 100 & 2.99937 & 6.99696 & 10.998\\
 200 & 2.99984 & 6.99933 & 11.0038 \\
 300 & 2.99978 & 6.99978 & 11.0049 \\
 400 & 2.99996 & 6.99993 & 11.0053 \\ \hline 
\end{tabular}
\caption{Beregnede egenverdier for $\rho_{max} = 4.5$}
\end{table}

Det ser ut til at $\lambda_2$ begynner å divergere for høye $n$. Det virker som 4.5 var en for lav verdi for $\rho_{max}$, så vi
prøver på nytt med $\rho_{max} = 5$, se Tabell 3.

\begin{table}[h!]
 \centering
 \begin{tabular}{|c|c|c|c|c|c|}\hline
  $n$ & $\lambda_0$ & $\lambda_1$ & $\lambda_2$ & Tid [s] & Antall iterasjoner \\ \hline
  100 & 2.99923 & 6.99617 & 10.9908 & 0.19 & 16475\\
  200 & 2.99981 & 6.99904 & 10.9978 & 2.94 & 66828 \\
  300 & 2.99991 & 6.99957 & 10.9991 & 14.4 & 150798 \\
  400 & 2.99995 & 6.99976 & 10.9996 & 44.66 & 269021 \\
  500 & 2.99997 & 6.99985 & 10.9998 & 108.05 & 422320 \\ \hline 
 \end{tabular}
\caption{Beregnede egenverdier for $\rho_{max} = 5$, sammen med tiden algoritmen brukte på å kjøre og antall iterasjoner
den brukte.}
\end{table}

Vi ser at $\lambda_0$ og $\lambda_1$ konvergerer omtrent like raskt for $\rho_{max}=5$ som for $\rho_{max}=4.5$, men
for $\rho_{max}=5$ divergerer ikke $\lambda_2$. For $n=500$ er resultatene veldig nøyaktige og kun med en liten feil i
fjerde desimal for $\lambda_1$ og $\lambda_2$. Vi legger også merke til at antall iterasjoner vokser kraftig med $n$.

\begin{figure}
 \centering
 \includegraphics[scale=0.7]{transformasjoner.png}
 \caption{Antall similærtransformasjoner som trengs før alle ikke-diagonale elementer omtrent er null som funksjon av
 $\rho_{max}$ og $n$.}
\end{figure}

Figur 1 viser antall similærtransformasjoner algoritmen bruker som funksjon av $\rho_{max}$ og $n$. Som funksjon av $\rho_{max}$
synker det langsomt med økende $\rho_{max}$, det kan se ut som et lineært forhold. Som funksjon av $n$ vokser det 
raskere og raskere for høyere $n$, forholdet ser eksponensielt ut her, eller i det minste kvadratisk. Dessverre er det altfor få datapunkter til å kunne
fastslå noen streng sammenheng.

For å undersøke dette nærmere plottes tiden algoritmen brukte mot antall punkter og antall tranformasjoner mot tiden
(Figur 2). Det ser ut til at antall transformasjoner er proporsjonal med logaritmen til tiden algoritmen bruker, mens
tiden algoritmen bruker ser veldig ut til å vokse eksponensielt med antall steg. Det er veldig rart at antall
transformasjoner og tiden algoritmen bruker ikke henger sammen lineært! Ut ifra plottene virker det i hvert fall
rimelig å anta
$$t\propto e^{n^\alpha}\Rightarrow \text{Transformasjoner} \propto \ln e^{n^\alpha} = n^\alpha $$
for en $\alpha>1$.

\begin{figure}
 \centering
 \includegraphics[scale=0.7]{tidsomfunkavn.png}
 \caption{Tid algoritmen bruker mot dimensjonen av matrisen og antall transformasjoner som må gjennomføres mot tiden. Utført ved
$\rho_{max}=5$.}
\end{figure}

For sammenlikning er har egenverdiene også blitt beregnet ved hjelp av formelen for å finne egenverdier av symmetriske matriser
i Armadillo: $eig\_sym$. Resultatene vises i Tabell 4. Det er hensynsmessig å sammenlikne resultatene med den analytiske løsningen.



\begin{table}[h!]
\centering
 \begin{tabular}{|c|c|c|c|c|}\hline
  $n$ & $\lambda_0$ & $\lambda_1$ & $\lambda_2$ & Tid [s] \\ \hline
  100 & 2.99923 & 6.99617 & 10.9908 & 0.00  \\
  200 & 2.99981 & 6.99904 & 10.9978 & 0.01 \\
  300 & 2.99991 & 6.99957 & 10.9991 & 0.03  \\
  400 & 2.99995 & 6.99976 & 10.9996 & 0.07  \\
  500 & 2.99997 & 6.99985 & 10.9998 & 0.13  \\ \hline 
 \end{tabular}
\caption{Beregnede egenverdier for $\rho_{max} = 5$ ved hjelp av Armadillo.}
\end{table}

Vi ser at Armadillo gir nøyaktig like resultater som Jacobis metode. Dette viser at Jacobis metode har blitt implementert riktig
ettersom Armadillo gir den nøyaktige numeriske løsningen på problemet. Ved å variere $\rho_{max}$ vil Armadillo og Jacobis metode
gi noe forskjellige svar. Da algoritmen ble kjørt med $\rho_{max}=8$ var Armadillo mer nøyaktig enn Jacobi. Dette tyder på at valget
av $\rho_{max}=5$ er det optimale for dette problemet. Legg også merke til Armadillo var vesentlig raskere enn Jacobis metode.

Videre ser vi på egenvektorene tilsvarende egenverdiene vi har funnet. Disse tilsvarer de numeriske løsningene for $u(r)$. Før disse 
plottes må de normaliseres. Dette gjøres ved en veldig enkel integrasjonsalgoritme i programmet som plotter resultatene.
De analytiske løsningene av den radielle Schrödingerlikningen er [2](med substitusjonen $\nu r^2=\f{1}{2}\rho^2$):
$$R_n(\rho) = N_n e^{-\f{1}{2}\rho^2}L_n^{\left(\f{1}{2}\right)}(\rho^2)$$

Hvor $L_n^{\left(\f{1}{2}\right)}$ er generaliserte Laguerrepolynomer og $N_n$ er en normeringskonstant.
Substitusjonen $r\rightarrow \rho$ medfører $\nu\rightarrow \f{1}{2}$, så
$$N_n^2 = \sqrt{\f{1}{4\pi}}\f{2^{n+3}n!}{(2n+1)!!}.$$

Dermed blir

\begin{align*}
 |u_0(\rho)|^2 &= \f{4}{\sqrt{\pi}}\rho^2e^{-\rho^2} \\
 |u_1(\rho)|^2 &= \f{8}{3\sqrt{\pi}}\rho^2\left(\f{3}{2}-\rho^2\right)^2e^{-\rho^2} \\
 |u_2(\rho)|^2 &= \f{8}{15\sqrt{\pi}}\rho^2\left(\rho^4-5\rho^2+\f{15}{4}\right)^2e^{-\rho^2} 
\end{align*}

\begin{figure}
 \centering
 \includegraphics[scale=0.7]{ettelektron.png}
 \caption{$u_n(\rho)$ for ett elektron i harmonisk oscillatorpotensial for $n=0,1,2$. Matrisedimensjonen er 500.}
\end{figure}

Figur 3 viser de numerisk beregnede bølgefunksjonene for både Armadillo og Jacobimetoden sammen med de analytiske
løsningene. Vi ser at alle løsningene stemmer godt overens med hverandre. Det er veldig heldig, for det tyder på at
algoritmene kan brukes med god nøyaktighet også når de analytiske funksjonene ikke er kjent.

Videre studerer vi to vekselvirkende elektroner i det tredimensjonale  harmoniske oscillatorpotensialet. I denne
oppgaven har Armadillo blitt valgt fordi den var mye raskere enn Jacobis rotasjonsmetode, men et par testkjøringer av
disse ga helt like resultater.

For tilfellene $\omega_r = 0.25$ og $\omega_r = 0.05$ er eksakte energier og analytiske bølgefunksjoner kjent [3]:
\begin{align*}
 \omega_r &= 0.25:\ \ \lambda = 1.25,\ \ \ u(\rho) = \rho e^{-\f{\rho^2}{8}}\left(1+\f{\rho}{2}\right)\\
 \omega_r &= 0.05:\ \ \lambda = 0.35,\ \ \ u(\rho) = \rho  e^{-\f{\rho^2}{40}}\left(1+\f{\rho}{2}+\f{\rho^2}{20}\right)
\end{align*}

(i notasjonen brukt i [3] er $\varepsilon'=\lambda/2$ og i polynomene har vi satt $\ell=0$).

Siden disse er kjente ble de brukt til å teste formelen for $\rho_{max}$ gitt over. Før plotting ble de analytiske funksjonene
normert over intervallet $[0,\rho_{max}]$.
Figur 4 viser de analytiske og numeriske løsningene i samme plott. Vi ser at kurvene ligger tett på hverandre, hvilket
betyr at valget av $\rho_{max}$ har vært heldig, spesielt siden andre valg av $\rho_{max}$ har gitt veldig dårlige
resultater, spesielt når $\rho_{max}$ ble for liten, men redusert presisjon har også forekommet når $\rho_{max}$ blir
for stor.

\begin{figure}[h!]
 \centering
 \includegraphics[scale=0.7]{tokjenteelektroner.png}
 \caption{De analytiske og beregnede bølgefunksjonene for to vekselvirkende elektroner i tredimensjonalt harmonisk
 oscillatorpotensial som funksjon av (den dimensjonsløse) avstanden mellom dem. Matrisedimensjonen er 800.}
\end{figure}


\begin{table}[h!]
 \centering
 \begin{tabular}{|c|c|c|}\hline
 $\omega_r$ & $\rho_{max}$ & $\lambda_0$\\ \hline
 0.01 & 102.1 & 0.105774 \\
 0.05 & 26.848 & 0.349999 \\
 0.25 & 10.2 & 1.25 \\
 0.5 & 7.85992 & 2.23011\\
 1 & 6.5937 & 4.05785 \\
 5 & 5.43144 & 17.4483 \\ \hline
 \end{tabular}
\caption{De utregnede verdiene av $\rho_{max}$ og $\lambda_0$ for de forskjellige $\omega_r$.}
\end{table}

Tabell 5 viser #include <iostream>
#include <cmath>
#include <ctime>
#include <armadillo>
#include <fstream>
#include <string>

using namespace arma;
using namespace std;


// Function for finding maximum off-diagonal element and placement (k,l)=(i,j) = (row, column) of that element.
// Call (double pointer to matrix, dimension, variable that returns as max element, -"- row, -"- column)

void MaxOffDiagonal(double ** A, int n, double &max, int &k, int &l){
    max = 0.0;
    for (int i = 0; i<n; i++){
        for (int j = i+1; j<n; j++){
            if (max < fabs(A[i][j])){
                max = fabs(A[i][j]);
                l = j;
                k = i;
            }
        }
    }

}

void FindSinAndCos(double a_kk,double a_ll,double a_kl,double &s,double &c){
    double t = 0;
    double tau = (a_ll-a_kk)/(2*a_kl);
    if (tau>=0){
        t = -tau+sqrt(1+tau*tau);
    }
    else{
        t = -tau - sqrt(1+tau*tau);
    }
    c = 1./sqrt(1+t*t);
    s = c*t;
}

void JacobiRotation( double**A, double **R, int n, int k, int l, double s, double c) {
    // Change matrix elements. All diagonal elements except ll and kk are conserved.
    // Change matrix elements with indices l and k:
    double a_ll = A[l][l];
    double a_kk = A[k][k];
    double a_kl = A[k][l];

    A[k][k] = a_kk*c*c - 2*a_kl*c*s + a_ll*s*s;
    A[l][l] = a_ll*c*c + 2*a_kl*c*s + a_kk*s*s;
    A[l][k] = 0.0;      // Forcing these to be zero.
    A[k][l] = 0.0;
    // Change remaining elements, new matrix will be symmetric:
    for (int i=0; i<n-1; i++){
        if (i!=k && i!=l){
            double a_ik = A[i][k]*c - A[i][l]*s;
            double a_il = A[i][l]*c + A[i][k]*s;
            A[i][k] = a_ik;
            A[i][l] = a_il;
            A[k][i] = a_ik;
            A[l][i] = a_il;
        }

        // Update eigenvectors:
        double r_ik = R[i][k];
        double r_il = R[i][l];
        R[i][k] = c*r_ik - s*r_il;
        R[i][l] = c*r_il + s*r_ik;

    }
}
// Function for finding the three smallest values of diagonal of matrix A, corresponding to the three lowest eigenvalues.
void ThreeMinValues(double ** A, int n, double &lambda1, double &lambda2, double &lambda3, int &vec0, int &vec1, int &vec2){
    lambda1=100; lambda2=100; lambda3=100;      // Cheating since we know them to be smaller than 100
    for (int i=0;i<n;i++){
        if (A[i][i]<lambda1){
            lambda1=A[i][i];
            vec0 = i;
        }
    }
    for (int i=0;i<n;i++){
        if (A[i][i]<lambda2 && A[i][i]>lambda1){
            lambda2 = A[i][i];
            vec1 = i;
        }
    }
    for (int i=0;i<n;i++){
        if (A[i][i] < lambda3 && A[i][i]>lambda2){
            lambda3 = A[i][i];
            vec2 = i;
        }
    }
}

// Test functions
void MaxOffDiagonalTest(){
    //Tests the function for finding maximum off-diagonal elements of a matrix.
    // Should return value (n-1)*(n-2) and placement [n-2][n-1], the function searches only upper triangle because of symmetry.

    int n = 5;
    double ** A;
    A = new double* [n];
    for(int i=0;i<n;i++){
        A[i] = new double[n];
    }
    for (int i=0; i<n; i++){
        for (int j=0; j<n; j++){
            A[i][j] = -i*j;

        }
    }
    int k, l;
    double max;
    MaxOffDiagonal(A, n, max, k, l);
    cout << "Max off-diagonal element test returned max = " << max << ", k = " << k << ", l = " << l << endl;
    cout << "Should return max = " << fabs(A[n-2][n-1]) << ", k = " << n-2 << ", l = " << n-1 << endl;
}

void ThreeMinValuesTest(){
    //Test the function for finding the three smallest diagonal elements of a matrix.
    // Should return values 1/(2n-1), 1/(2n
    int n = 5;
    double ** A;
    A = new double* [n];
    for(int i=0; i<n; i++){
        A[i] = new double[n];
    }
    for(int i=0; i<n; i++){
        for (int j=0; j<n; j++){
            if(i==j){
                A[i][j] = 1./(i+j+1);
            }
            else{
                A[i][j] = 0.0;
            }
        }
    }
    double lambda0, lambda1, lambda2;
    int k0, k1, k2;
    ThreeMinValues(A, n, lambda0, lambda1, lambda2, k0, k1, k2);
    cout << "Three minimum diagonal elements test returned values: " << endl;
    cout << lambda0 << "    " << lambda1 << "      " << lambda2 << endl;
    cout << "and placements: " << endl;
    cout << k0 << "     " << k1 << "      " << k2 << endl;
    cout << endl;
    cout << "Should return values: " << endl;
    cout << A[n-1][n-1] << "    " << A[n-2][n-2] << "    " << A[n-3][n-3] << endl;
    cout << "and placements: " << endl;
    cout << n-1 <<  "       " << n-2 << "       " << n-3 << endl;
}

int main()
{
//    MaxOffDiagonalTest();
//    ThreeMinValuesTest();
    int n = 801; //Dimension of matrices +1 and number of points in interval [0, rho_max]


    double epsilon = 1e-8;  // Error tolerance in off-diagonal elements ca. 0


    double omega_r = 1; // 'Frequency' parameter

    double rho_max = 1./pow(2*pow(omega_r, 2),1./3)+4/(5*omega_r)+5;          // Cutoff \approx infinity


    double h = rho_max/n;  // Step length
//    cout << rho_max << endl;


    double * rho;
    rho = new double[n-1];
    // Set up rho-values
    for (int i=0; i<n-1; i++){
        rho[i] = (i+1)*h;
    }

    // Set up eigenvector matrix, initially diagonal


    //double R[n][n];
    double ** R;
    R = new double*[n-1];
    for(int i=0; i<n-1; i++){
        R[i] = new double[n-1];
    }

    for (int i = 0; i<n-1; i++){
        for (int j=0; j<n-1 ; j++){
            if (i==j){
                R[i][j] = 1.0;
            }
            else {
                R[i][j] = 0.0;
            }
//        cout << R[i][j] << " ";
        }
//        cout <<    "    " << endl;
    }



// Set up matrix to find eigenvalues for.
    double e = -1./(h*h);   //Off-diagonal elements

    double const_in_d_i = -2*e;    //Because d[i] = 2/h^2+rho[i]^2, no need to compute constant term n times.

/*
    double * d;
    d = new double[n-1];

    for (int i=0; i<n-1; i++){
        d[i] = const_in_d_i + omega_r*omega_r*rho[i]*rho[i]+1./rho[i];    // diagonal elements, potential
    }




    // Set up matrix to be Jacobi rotated:
    double ** A;
    A = new double* [n-1];
    for(int i=0;i<n-1;i++){
        A[i] = new double[n-1];
    }

    for (int i=0; i<n-1; i++){
        for (int j=0; j<n-1; j++){
            if (i==j){
                A[i][j] = d[i];
            }
            else if (fabs(i-j)==1){
                A[i][j] = e;
            }
            else {
                A[i][j] = 0.0;
            }
//            cout << A[i][j] << "    ";
        }
//        cout << "   " << endl;
    }



    // Values for max off-diagonal element in A:
    double max=1.0;
    int k;
    int l;
//    clock_t start, finish;      //set up timer
//    start = clock();

    int iteration_counter = 1; //Counting no. of iterations, starting at 1 since we do one before the loop.

    MaxOffDiagonal(A, n-1, max, k, l);      // Do this once before the loop, so that it terminates once max>epsilon and not goes another round
    while (max>epsilon){
        double a_ll = A[l][l];
        double a_kk = A[k][k];
        double a_kl = A[k][l];

        double c=0;
        double s=0;    // For values of cot, tan, cos and sin of the angle of rotation

        FindSinAndCos(a_kk,a_ll,a_kl,s,c);
        JacobiRotation(A, R, n, k, l, s, c);
        MaxOffDiagonal(A, n-1, max, k, l);
        iteration_counter++;

    }

*/

//    string file_name_arm = "Eigenvectors_armadillo_n" + to_string(n) + ".txt";

//    string file_name_arm = "Eigenvectors_armadillo_omega_r" +to_string(omega_r) + ".txt";

//    string file_name_arm = "Eigenvectors_nonint_omega_r" +to_string(omega_r) + ".txt";


//    ofstream outfile_arm;
//    outfile_arm.open(file_name_arm.c_str());



    // Armadillo for comparison
    mat B(n-1,n-1, fill::zeros);
    vec D(n-1, fill::zeros);

    D += const_in_d_i;
//For interaction:
/*
    for (int i=0;i<n-1;i++){
        D(i) += rho[i]*rho[i]*omega_r*omega_r + 1./rho[i];
    }
*/
//Non-interacting:
    for (int i=0;i<n-1;i++){
        D(i) += rho[i]*rho[i]*omega_r*omega_r;
    }

    B.diag()+=D;
    B.diag(1) += e;
    B.diag(-1) += e;

    vec eigval;
    mat eigvec;

    eig_sym(eigval, eigvec, B);

//    cout << eigval(0) << endl;
//    cout << eigval(1) << endl;
//    cout << eigval(2) << endl;

//    cout << "       " <<endl;

//    for(int i=0;i<n-1;i++){
//        outfile_arm << rho[i] <<"       " << eigvec(i,0) << "     " << eigvec(i,1) << "       " << eigvec(i,2) << endl;
//    }

//    finish = clock();
/*
    double lambda1, lambda2, lambda3;   // Eigenvalues
    int vec0, vec1, vec2;               // Placement of eigenvectors
    ThreeMinValues(A, n-1, lambda1, lambda2, lambda3, vec0, vec1, vec2);

    cout << vec0 << "   " << vec1 << "  " << vec2 <<endl;
    // Open file to write out eigenvectors for lambda_0, lambda_1, lambda_2


    string file_name = "Eigenvectors_n" + to_string(n) + ".txt";

    ofstream outfile;
    outfile.open(file_name.c_str());

    for (int i=0; i<n-1; i++){
        outfile << rho[i] << "  " << R[i][vec0] << "     " << R[i][vec1] << "   " << R[i][vec2] << endl;
    }
*/

//    cout << lambda1 << endl;
//    cout << lambda2 << endl;
//    cout << lambda3 << endl;

//    cout << ((double) (finish-start)/CLOCKS_PER_SEC) << endl;
//    cout << iteration_counter << endl;
    return 0;
}


de beregnede grunntilstandsenergiene for to-elektronsystemet. For de kjente verdiene av $\omega_r$ ser
vi at resultatene er gode. Dette betyr antakelig også at de er gode for $\omega_r=0.01$ siden den er av samme størrelsesorden, 
men det er umulig å si sikkert. Det er tilsvarende umulig å si noe sikkert om de større verdiene av $\omega_r$.


\begin{figure}[!htb]
 \centering
 \includegraphics[scale=0.7]{toelitenomega.png}
 \caption{Den beregnede radielle bølgefunksjonen til to vekselvirkende elektroner i tredimensjonalt
 harmonisk oscillatorpotensial for $\omega_r=0.01$ og $\omega_r=0.5$.}
\end{figure}

\begin{figure}[!htb]
 \centering
 \includegraphics[scale=0.7]{toestoromega.png}
 \caption{Den beregnede radielle bølgefunksjonen til to vekselvirkende elektroner i tredimensjonalt
 harmonisk oscillatorpotensial for $\omega_r=1$ og $\omega_r=5$.}
\end{figure}

\begin{table}[h!]
 \centering
 \begin{tabular}{|c|c|}\hline
  $\omega_r$ & $\lambda_0$\\ \hline
  0.01 & 0.0299995\\
  0.5 & 1.49999\\
  1 & 2.99998\\
  5 & 14.9996\\ \hline
 \end{tabular}
\caption{$\lambda_0$ beregnet for forskjellige $\omega_r$ i det ikke-vekselvirkende tilfellet. $\rho_{max}$ er i hvert tilfelle
det samme som i Tabell 5.}
\end{table}

Tabell 6 viser grunntilstandsenergiene i det ikke-vekselvirkende tilfellet, hvor $\rho_{max}$ har blitt beregnet
med samme formel som før.

Figur 7 og 8 viser de beregnede bølgefunksjonene for de to ikke-vekselvirkende elektronene.

\begin{figure}[!htb]
 \centering
 \includegraphics[scale=0.7]{ikkevvlitenomega.png}
 \caption{Den beregnede radielle bølgefunksjonen til to ikke-vekselvirkende elektroner i tredimensjonalt
 harmonisk oscillatorpotensial for $\omega_r=0.01$ og $\omega_r=0.5$.}
\end{figure}

\begin{figure}[!htb]
 \centering
 \includegraphics[scale=0.7]{ikkevvstoromega.png}
 \caption{Den beregnede radielle bølgefunksjonen til to ikke-vekselvirkende elektroner i tredimensjonalt
 harmonisk oscillatorpotensial for $\omega_r=1$ og $\omega_r=5$.}
\end{figure}


%\end{multicols*}
\pagebreak

\newpage
\section{Diskusjon}
Det virker som om metoden brukt for å finne egenverdier til bølgelikningen er robust. Ettersom de kjente verdiene
for det vekselvirkende tilfellet ble reprodusert, men også på grunn av de beregnede verdiene for det ikke-vekselvirkende tilfellet:
Ettersom bølgelikningen reduseres til
$$\left(-\f{d^2}{d\rho^2}+\omega_r^ 2\rho^2\right) u = \lambda u.$$

Dette kan sammeliknes med (den dimensjonsfulle) likningen for ett elektron:

$$\left(-\f{d^2}{d\rho^2}+\f{m^2\omega^2}{\hbar^2}\rho^2\right) u = \f{2mE_{nl}}{\hbar^2} u $$

med energiløsningene: $E_{nl} = \hbar \omega (2n+\ell +3/2)$. Her kan vi bare sette $\omega_r = \f{m\omega}{\hbar}$
og $\lambda = \f{2mE_{nl}}{\hbar^2}$, helt analogt med hvordan vi forenklet ett-elektronlikningen.
For $n=\ell=0$ får vi da:

$$\lambda = \f{2mE_{00}}{\hbar^2} = \f{2m\omega}{\hbar}\cdot \f{3}{2} = 3\omega_r$$

hvilket passer perfekt med tabell 6! Dette tyder også på at valget av $\rho_{max}$ har vært godt, men det hadde vært lurt
å kjøre for forskjellige verdier her for å sjekke hvilken betydning valg $\rho_{max}$ har her.

I plottene av bølgefunksjonene har noe rart skjedd: de her helt like med og uten vekselvirkning. Det hadde vært lurt
å plotte de analytiske løsningene for det ikke-vekselvirkende tilfelle til sammenlikning, men det ble det dessverre ikke
tid til. Dette sår tvil i nøyaktigheten til de beregnede bølgefunksjonene i det vekselvirkende tilfellet selv om de passet
veldig bra med de analytiske løsningene der slike fantes.

Vi kan uansett analysere hovedtrekkene ved plottene: jo høyere $\omega_r$ jo nærmere $\rho = 0$ havner toppunktet
i sannsynlighetsfordelingen for posisjonen til elektronene i forhold til hverandre. Dette er som forventet ettersom
koplingskonstanten mellom elektronene ikke endrer seg, så den frastøtende kraften mellom den er like stor mens
potensialet som hindrer dem fra å dra uendelig langt fra hverandre øker.

Observerer også at toppunktet til sannsynlighetsfordelingen ligger lavere jo lavere $\omega_r$ er. Det vil si at fordi
lavere $\omega_r$ er bølgefunksjonen smurt tynnere ut over rommet og posisjonen er mindre skarp, mens for høyere
$\omega_r$ blir posisjonen skarpere.
Dette gir mening ettersom det burde være usannsynlig å finne elektronene i en posisjon hvor potensialet er stort
og ved samme argumentasjon som over skjønner vi at området hvor potensialet er ``tilstrekkelig lite'' blir mindre etterhvert
som $\omega_r$ øker. Når $\omega_r$ minker blir dette området større og i tillegg blir potensialet veldig nært 0
over en større del av rommet, slik at sannsynligheten blir jevnere fordelt over et større område.

Det hadde vært lurt å beregne alle bølgefunksjonene for samme verdi av $\rho_{max}$ for å bedre kunne sammenlikne dem.

Angående de numeriske metodene som har blitt brukt virker Armadillo knusende overlegen. Med mindre det finnes
spesielle tilfeller hvor Jacobis metode konvergerer raskere enn den gjør her, eller en implementering av den som
kjører mye raskere er det vanskelig å se verdien i den. Annet enn å lære seg å programmere, så klart.

\section{Kilder}


1: http://web.stanford.edu/class/cme335/lecture7.pdf


2: https://en.wikipedia.org/wiki/Quantum\_harmonic\_oscillator

3:  M. Taut, Phys. Rev A 48, 3561-3566 (1993)

\end{document}
